\documentclass[conference]{IEEEtran}
\usepackage{cite}
\usepackage{amsmath,amssymb,amsfonts}
\usepackage{algorithmic}
\usepackage{graphicx}
\graphicspath{ {./img/} }
\usepackage{textcomp}
\usepackage{xcolor}
\usepackage{array}
\def\BibTeX{{\rm B\kern-.05em{\sc i\kern-.025em b}\kern-.08em
    T\kern-.1667em\lower.7ex\hbox{E}\kern-.125emX}}
\begin{document}

\title{DOOWA\
\thanks{}
}

\author{\IEEEauthorblockN{Wan}
\IEEEauthorblockA{\textit{Dept. of Computer Software} \\
{Hanyang University}\\
Kuala Lumpur, Malaysia \\
aus.baik@gmail.com}
\and
\IEEEauthorblockN{Amin}
\IEEEauthorblockA{\textit{Dept. of Computer Software} \\
{Hanyang University}\\
Kuala Lumpur, Malaysia \\
amin.sharudin@gmail.com}
\and
\IEEEauthorblockN{Megat}
\IEEEauthorblockA{\textit{Dept. of Information Systems} \\
{Hanyang University}\\
Johor, Malaysia \\
megattmf@gmail.com}
}

\maketitle

\begin{abstract}
After living in the COVID-19 pandemic for approximately 2 years now, we decided that we want to make something that can alleviate people suffering during this pandemic. We came up with an Android mobile application called ‘DOOWA’. The name DOOWA is derived from the Korean word ‘\textit{deobda}’ which means to help. The main function of our application is to connect individuals or families in the community who required assistance to survive during this pandemic to a donor in the community who are willing to lend a helping hand to reduce their hardship. We are hoping that this app will facilitate and encourage members of the community to help each other during this troubling times.

\end{abstract}
\begin{IEEEkeywords}
DOOWA, Android Application, Assistance, Donation
\end{IEEEkeywords}

\section{Introduction}
\subsection{Motivation}

    We focused on people who are financially less fortunate  to be seen by the community. We created this mechanism to have a balanced community and people can reach out and seek help from others easily. There are families that are hugely affected, and many suffer from Covid-19. As much as we would like to get rid of any bad situations, it is still unavoidable and usually unexpected. The nature and consequences of these situations can vary significantly and in worst cases can also be life threatening. Therefore, we think that it would be nice to have some mechanism by which we can notify and get notified by certain people about such circumstances and increases the chances of giving and receiving help as soon as possible.
\\ \\
\indent The need for such a mechanism increases even more as in this era of technology, platforms exist to support them. One such platform and a very common one in that a mobile application. Almost everyone today has an access to mobile app as they are easy to use and can be accessed by phone and tablets. Hence, this motivates our team to develop a mobile application for giving and receiving help in the community.
\\
    
\subsection{Problem Statement}

-	Donor receivers are difficult in reaching out for help as they do not have access with other donors physically.\\

-	Some receivers are not able to find the type of donations that they really need. As a result, the donation given is wasted.\\

-	It is hard to contact between donors and receivers as they do not know each other before.\\

- 	We acknowledge that people in need can also request help from the government, but often times this help have to go through a long, bureaucratic application process before they are approved.\\


\begin{table}[htbp]
\caption{Role Assignments}
\begin{center}
\begin{tabular}{ | m{5em} | m{2cm}| m{3cm} | } 
  \hline
 \textbf{Roles}& \textbf{Name} & \textbf{Task Description} \\
\hline
  \textbf{User/ Customer} & Megat & Act as a beta tester to criticize and provide suggestions from a client point of view. They will provide how the application UI should look like to satisfy the ease of access from a user perspective. They also should test every feature in the app to identify bugs and report to the software developer. \\
  \hline
  \textbf{Software Developer} & Wan & The software developer should have the general point of view of how the application works overall. They will provide the services for backend server and database as well as providing the general UI of the frontend.   \\ 
  \hline
  \textbf{Development Manager} & Amin & Development manager will be the main overseer of the project development, and gathers the information from the client side and handles the reports. They also will be the main proofreader for the project documentation to meet the project specification.	\\
  \hline
\end{tabular}
\end{center}
\end{table}

\subsection{Research on any related software }

\begin{enumerate}
\item \textit{KakaoPay }\\ 
KakaoPay is a mobile payment and digital wallet service by Kakao based in \indent South Korea that allows users make mobile payments and online transactions easily. It ensures smooth operations where users can make payments or do a money transfer via KakaoPay handily and there is no floating time for recipients to receive the money. It also lets users to invest with a small amount of money, get a loan for a house, and find the perfect insurance partner. Users can also save any credit or debit card information on it so that they can make one-time payment easily without filling in payment information once again with just only one tap. The service also supports contactless payments where users can send an amount of money to anyone they want to. Users can also notify recipients if they have successfully transferred the money and vice versa. \\
\item \textit{PayPal} \\
PayPal is one of the world’s largest payment services that is secured with advanced technologies. PayPal offers a worldwide payment service and supports Visa, MasterCard and so on. Users can sign up for PayPal account to have an extra level of security and fraud prevention with a quicker payment option and save payment details for future transactions. PayPal also lowest transaction fees for a global transaction, therefore users can freely use any card they prefer to use. Besides, it also offers reward points for each successful transactions that can later benefit users to transfer money wirelessly with even lower transaction fees. PayPal also has its own digital wallet that users can put money in so that users can directly transfer money without entering one’s account numbers every single time.\\
\item \textit{Yogiyo }\\
Yogiyo is a food delivery service application which enable users to get their food delivered at their doorstep from various restaurants easily. The application connects users to a variety of restaurants from different cuisines such as Western, Korean, Japanese and Chinese. One of the features of the application that we want to emulate is their delivery tracking feature. Through the application users can know the location of the delivery food rider in real time. Rider’s location is represented by an icon on the map in the application, and the icon moves in relation to the rider’s location. This is the feature that we want to have in our application so that whenever a meeting is set up, one person can know the location of the other party in real time and plan accordingly.\\
\item \textit{Coupang }\\
Coupang is an online shopping application based in Seoul, South Korea that sells products from a wide-ranging category including food, clothing, fresh produce, baby products and many more. User can shop online in the comfort of their own home and have the products delivered on their doorsteps. In the application, users can know the current location of their parcel through an icon. Every time the parcel moves from the seller to the warehouse or currently in delivery, each and every stage of this process is shown to the user so that they can have the assurance that the product that they buy will arrive. We want to do the same thing with our application in case a donor wants to send products through the postal service. \\
\item \textit{Google/Facebook Account}\\
Google and Facebook account is something that the majority of people have. We have also seen a lot of application nowadays which requires first time users to make an account if they want to use the services provided by the application. Similar to this, our application will also require first time user to register an account with us. However, instead of filling in their details one by one, we will allow users to use their already existing Google of Facebook account to register on our application. This will ensure smooth registration process and provide a hassle-free service to our users.\\
\item \textit{Sambal SOS App} \\
The effects of pandemic have hit Malaysia in many aspects that this has increased the suicide rates at an all-time high. Therefore, Sambal SOS app or know as “White Flag” is an application that is programmed by three Malaysian students where the purpose of the application is to connect Malaysians who are in need with those who can help. Users can also access through web application and log in with Google or Facebook account. This application is also implemented map where users can give their hands easily depending on their location. Donator and recipient can immediately connect on this app to have further details regarding their needs. Recipient can also fill in their details and what type of donations they need so that donator can have detailed information before giving out donation.  This application has a basic interface where all generations are able to use the features on the application at ease. \\
\item \textit{SirenGPS} \\
SirenGPS integrates emergency management tools with real-time visibility and interoperability for emergency managers, first responders, and stakeholders in your community. This application also has a Siren Alert feature that can send real-time messages to specific groups, locations, buildings or your entire community. Siren Alert not only allows you to inform individuals of a crisis in their immediate vicinity, it lets you warn people as they approach a threat and steer them toward safety. With Siren Alert, you can also enable individuals in your community to respond back in real-time. Our application plans to integrate the Siren Alert feature and enable anyone to alert any person nearby for donation. \\
\item \textit{ShareTheMeal}\\
This application is launched by the United Nations with the aim of feeding the needy especially with foods. This app is available in about 9 languages, including English, Spanish, French, German, Italian, Portuguese, Russian, Korean, and Japanese. Besides, this app gives users a direct channel to fund UN efforts in supporting children in need, on the go, with just a single tap. It only costs 0.50USD to feed one hungry child a day, plus users can track where the meals are distributed and the impact of the donations to the needy. ShareTheMeal application is an extension of the ShareTheMeal non-profit initiative run by the World Food Programme (WFP) which is the world’s largest humanitarian agency combating hunger globally and providing food aid to an estimated 80 million people each year. Therefore, this application covers a wide range of location where people in need get to seek help easier.\\
\end{enumerate}


\section{Requirement}

\subsection{Functional Requirements}

\begin{itemize}
\item \textbf{\\Account Creation}
\end{itemize}
\par The application should have an account creation system for the users before enabling them to access all the features on the app. On the main screen, there should be a button for sign up option. User will have to enter their email and password and these data will be stored in backend database. In addition, all email address should be unique and users cannot sign up with an already used email address. There should be no specific requirement for password but the application should display the strength of the password inputted by user. Upon completing the registration, a verification email should be sent to the new users’ email address. All users will also have unique ID bind to their account.\\
\begin{itemize}
\item \textbf{Login}
\end{itemize}
\par Upon opening the application, the main screen should display the login menu with blank space for the users to input their email address and password. After the users entered their credentials, the inputted info will be sent to the backend server and it will search for matching email address and password combination in the database. In case of successful login, the application will inform the user that the login was successful and the user will be sent to the main page of the application. In failure case, the user will be rejected by the system and prompted to enter the correct combination.  Alternatively, the application will provide the users another option to login via Google, Facebook or Twitter.\\
\begin{itemize}
\item \textbf{Chat Bot}
\end{itemize}
\par The application should provide a chat bot service to the users which let the users to ask questions and set a reminder. This chat bot should not use an AI approach, bdut rather a very simple approach where the users just type in the command and arguments and the bot will reply accordingly. The chat bot should have various kind of features and commands that covers all the users’ needs.\\
\begin{itemize}
\item \textbf{Maps}
\end{itemize}
\par The applicadftion should have a map as the main interface. To implement this, the Google Maps API \\services will be used. On this map, it should show all the locations of the otherd users that are requesting for donation and nearby foodbanks. If the user is a donor, the map should show the location of where the donor is giving out the donation, the time available and the type of donation which will be indicated by an icon on the map. If the user is a receiver, the map will display their location for another donor to come and visit them and the severity of their desperation which will be indicated by color hue of their location. The locations of the two types of users should be indicated by the common map pin with different color each. If the location is an organization type donor, a special pin should be used.\\
\begin{itemize}
\item \textbf{Online Banking}
\end{itemize}
\par Donation can be made by physically or online transfer for money. The application should provide an online banking service for the users which let online money transfer between the users. Services such as Paypal, FPX banking and KakaoPay should be implemented.\\
\begin{itemize}
\item \textbf{Points for Donor}
\end{itemize}
\par Setting up points for donor every time they made a donation. This kind of grading system will allow us to identify frequent donors who will be rewarded accordingly with coupons based on their numbers of contribution. Donors will also be put into groups/level according to the points they accumulated.\\
\begin{itemize}
\item \textbf{Notifications}
\end{itemize}
\par The notification system in the app will pop-up whenever the help that you requested are answered by a donor. This will help the requester to be alert at all time.\\
\begin{itemize}
\item \textbf{Donor Rating}
\end{itemize}
\par The app will have a rating and review system for its donors and aid receivers. An aid receiver can rate their donor based on a 5-star rating system and also leave a review after they have received the help that the required. This system is put in place in order to prevent scams, and accounts who does not follows our community guidelines.\\
\begin{itemize}
\item \textbf{Reminder}
\end{itemize}
\par The app will also have a notification system where you can set, to remind you of the date and place of meeting to receive or deliver your donation. You can set the notification to pop up at 2 hours or 30 minutes before your meeting time.\\
\begin{itemize}
\item \textbf{Chat Room between Donor/Receiver}
\end{itemize}
\par The chat function between the donor and the receiver is important so that they can get more detailed information on what kind of help that they require. They can also use the chat function to set up time and place to meet if necessary.\\
\begin{itemize}
\item \textbf{Donor Tracker (postage/meeting)}
\end{itemize}
\par We believe that any form of help is important and should be facilitated. Therefore, in a case where the donor wanted to send the required items through postage, we will have a function where the app tracks the position of your postage. This will help the receiver to identify the location of the items that they so direly need at all times. This function can also be modified to tract the location of individuals on the way to the meeting spot which they have agreed upon beforehand.\\
\begin{itemize}
\item  \textbf{Set a Face-to-Face Meeting}
\end{itemize}
\par This allows donators to have face-to-face meetings with the recipients within a specific time. 
\begin{enumerate}
\item  Click the “Time” to pop out the clock to set the meeting time.
\item  Click the “Date” to pop out the calendar to set the meeting date and day.
\item  Past dates cannot be clicked as there will be an error message.
\item  The furthest appointment that can be made is one month ahead. Hence, user cannot make an appointment that is more than one month ahead.
\item There are no multiple appointments in one meeting. Therefore, if a date has been chosen for a meeting, user cannot reserve any other dates before cancelling the previous chosen date.
\item The dates must be agreed by both donator and the recipient. Click “Accept” button to agree the dates and “Reject” button to inform that one is not available for the chosen date.
\item  If both parties agreed to the chosen dates, they then move to the confirmation page. \\
\end{enumerate}
\begin{itemize}
\item \textbf{Choosing Types of Donation}
\end{itemize}
\par This allows donators to categorize their types of donations so that it is easier for recipients to seek for help with a specific type of donation.
\begin{enumerate}
\item Donators and recipients can choose which type of goods is important to them. This includes money, clothing, foods, baby items, gadgets, home essentials and others. 
\item  If Others is picked, user can state the type of donation that will be give or needed. 
\item Once the type of donation is picked, then it will only show donator/recipient that have the same type of donation. 
\item This will help donators and recipients to discover those who have the same interest. \\
\end{enumerate}
\begin{itemize}
\item \textbf{Pictures Upload}
\end{itemize}
\par This allows users to upload pictures of themselves and the donations for confirmation and proof so that it will make the donation more precise and clearer.
\begin{enumerate}
\item  Click ‘Camera’ button to connect the application to the camera.
\item  The pictures taken are going to be stored and shared in the user’s profile for others to confirm them.
\item  Click ‘Trash Bin’ button to delete pictures on the profile and there will be a pop-out whether to confirm the deletion.
\item  Click ‘Sure’ for confirm deletion, and ‘Cancel’ to cancel the deletion request. \\
\end{enumerate}
\begin{itemize}
\item \textbf{Guidelines and Manuals}
\end{itemize}
\par There is a guideline check for donators and recipients to follow to prevent any misbehavior and illegal actions throughout the donation process. This helps users to use the application properly.
\begin{enumerate}
\item  Click ‘Guidelines’ to view the guidelines of the application that help users to understand better about the application operation. 
\item  For first-time user, click ‘Agree’ to accept and follow the guidelines given. 
\item  Click ‘Disagree’ if the guidelines if the user does not accept the guidelines of the application. \\
\end{enumerate}
\begin{itemize}
\item \textbf{Report Button}
\end{itemize}
\par All entries from donor and receivers should comply with the terms and agreements of the application. Hence, for every entry, a report button must be provided. Whenever a user sees any other donor/receiver entry that does not comply with the guidelines, the user can use the report the button to alert the developers. This report feature should provide text box for what kind of terms violated. The report will be sent to the backend database and will be reviewed by the developers. Every report should have a unique ID and the severity point of the violation. The developer then will review the reports starting from the entry with the most severity points.  \\
\begin{itemize}
\item \textbf{Terms and Conditions}
\end{itemize}
\par This will provide legal agreements between us and people who are using our application. Users must agree to abide the terms of the service (application) to use it. This can also act as a disclaimer for users to understand and aware of the terms and conditions that are applied.
\par Users must acknowledge that the application is collecting names, addresses, credit card information or other personal data from the users. The data given may be used, stored, and shared. Besides, any misbehavior and illegal actions would result in termination of the user accounts. 
\begin{enumerate}
\item  Before logging in, user is opted to read and understand all the terms and conditions in the application.
\item  User must scroll through all the terms and conditions before clicking the ‘Agree’ button. 
\item  ‘Agree’ button will pop out if the user has scrolled through the terms and conditions.
\item  Click ‘Disagree’ button if user does not accept the terms and conditions. 
\item  If the user clicked ‘Disagree’ button, user is unable to log in and eventually use the application.\\
\end{enumerate}

\subsection{Non-Functional Requirements}
\begin{itemize}
\item  The application should have dark mode UI.
\item  The application should ask for satisfaction rating periodically
\item  The application should be simple enough to be accessible to elder users.
\item  The application must detect the internet connection before giving access to the users.
\item  The application should have an option to change language between English and Korean.
\item  The application must be ready before the final week of class. 
\end{itemize}

\section{Development Environment}
\subsection{Choice of Development Platform\\}

1) Platform used\\

\par We have decided to use the latest Windows 10 and MacOS as our main development platform for this project. Windows 10 is a good choice for coding because it supports many programs and languages. In addition, it has significantly improved over other versions of Windows and comes with various customization and compatibility options. There are also many advantages to coding on Windows 10 over Mac or Linux. It also provides great security features, easy to upgrade and supports a huge range of programming language such as PHP, Android and XML which are some of the language that will be the main backbone of our program development. On the other hand, MacOS is a Unix-based operation system and is popular choice nowadays for programming.  Programmers who work on a lot of back-end web server code often like MacOS for their personal computer, because it's based on Unix and easily runs nearly all Linux software.\\

2) Programming Language\\
\begin{itemize}
\item Java\\
\end{itemize}
\par We have decided to use Java as our main programming language for our project development; about 70 percent of it. We have chosen Java because first of all, the official language for Android development is Java. Large parts of Android are written in Java and its APIs are designed to be called primarily from Java. Plus, every beginner programmers, including us already have some experience in creating programs using this language. Moreover, this language is object-oriented which we think is for easier to use for developing an Android application since Java provides a various choices of GUI interface services. Moreover, Java provides a lot of libraries that can be used a lot for our program development such as Listeners, Fragments, Events, and much more. This language will be used for the frontend side of our program which handles the UI action such as the buttons actions after you clicked them or how the input by the users will be handled.\\
\begin{itemize}
\item XML\\
\end{itemize}
\par Extensible Markup Language (XML) is a great choice to be used in developing our application interfaces. We decided to use XML mainly because it is the main language that is being used by the Android Studio for UI designing and various widgets. Moreover, XML is designed to store and transport data and organize its data as a structure. Many UI frameworks use XML as the language to make the UI and once  you understand values and attributes, you know enough to look at an XML file and understand the data within.\\
\begin{itemize}
\item PHP\\
\end{itemize}
\par We have decided to use PHP as our main programming language for the backend server to interact with the database. PHP is a popular general-purpose scripting language that is especially suited to web development. The reason why we are choosing PHP for this project is because our backend server will be communicate with the database through the Internet and we need to provide a universal domain which enables our application to fetch data from database while being connected to any type of Internet connection. Moreonver, there are a lot of ready made PHP code that provides various services which would be helpful to be integrated into our program.\\
\begin{itemize}
\item SQL\\
\end{itemize}
\par We will use SQL (Struted Query Language) for managing our database.  It is a domain-specific language used in programming and designed for managing data that are in our relational database management system. It is particularly useful in handling structured data so this makes it possible to process data such as user accounts, donation requests, list of foodbanks, and map markers.\\

3) Cost Estimation

\begin{table}[htbp]
\caption{Cost Estimation}
\begin{center}
\begin{tabular}{ | m{5em} | m{4cm}| m{2cm} | } 
  \hline
 \textbf{Roles}& \textbf{Description} & \textbf{Cost} \\
\hline
  \textbf{Android Studio} & Frontend Code Editor & 0\\
  \hline
  \textbf{Visual Studio Code} &  Backend Code Editor & 0	\\
  \hline
  \textbf{Github} & Remote repository& 0\\ 
  \hline
  \textbf{Heroku} & Backend Server & 0	\\
  \hline
  \textbf{MySQL} & Backend Database & 0	\\
  \hline
  \textbf{Overleaf} & Documentation typesetting program & 8 USD/month\\
  \hline
  \textbf{Google Meet} &Video Conference software& 0	\\
  \hline
  \textbf{KakaoTalk} &Chat application& 0	\\
  \hline
\end{tabular}
\end{center}
\end{table}

4) Information of Development Environment\\
\begin{itemize}
\item Android Studio
\end{itemize}

\par \begin{figure}[h!]
\includegraphics[width=3cm]{android studio}
\centering
\end{figure}Android Studio provides the fastest tools for building apps on every type of Android device. For frontend, we can create complex layouts in a short amount of time because the layout can be designed by using the drag and drop feature provided by the IDE which will create the code by itself with zero compile error. After designing, the programmer can immediately see the result of the interface and how it will be look on our phone screen. Furthermore, it provides a fast and reliable Android phone simulator that can be used to preview your application on the phone right after the code is compiled. With this, we do not have to build the APK file and install it onto our phones every time we want to test the application. Finally, Android Studio can be run in Windows and MacOS which is perfect for our development environment.\\
\begin{itemize}
\item Visual Code Studio
\end{itemize}
\par \begin{figure}[h!]
\includegraphics[width=4cm,height = 3cm]{VSC}
\centering
\end{figure} We decide to use Visual Code Studio as the source code editor for PHP and SQL which will be handled by the backend because they have a built-in support and plugins for both of the programming language. IT also have the auto compile feature which can ease the process of testing our application without having to compile every time the code is edited. \\
\begin{itemize}
\item GitHub
\end{itemize}
\par \begin{figure}[h!]
\includegraphics[width=4cm]{github}
\centering
\end{figure} Github is one of the most popular website to upload our sources code seamlessly as well as share and develop the program with all the teammates. Moreover, our teammates can easily check the commits done by other members and implement necessary update of the features. Github provides necessary management functions for software development such as basic functions of the Git which includes functional requests, task management, and bug tracking.\\
\begin{itemize}
\item Heroku
\end{itemize}
\par \begin{figure}[h!]
\includegraphics[width=4cm]{heroku}
\centering
\end{figure} The reason why we choose Heroku as our backend server is because it provides a lot of services for a free price such as modules and ClearDB service which will be used as the direct connection between the server Heroku and the MySQL database. Furthermore, our team members are more well versed with the Heroku services which would speed up our application development much faster.\\
\begin{itemize}
\item MySQL
\end{itemize}
\par \begin{figure}[h!]
\includegraphics[width=4cm]{MySQL-Logo}
\centering
\end{figure}We will use MySQL for data management because as mentioned before, it can be directly connected with our Heroku server. Today, MySQL is the second ranking RDBMS solution in the world, according to DB Engines. Its users include a wide range of websites and applications, including household brands like Spotify, Netflix, Facebook and Booking.com. Finally, the data stored there can be easily viewed in tables.\\
\begin{itemize}
\item XAMPP
\end{itemize}
\par \begin{figure}[h!]
\includegraphics[width=4cm]{xampp}
\centering
\end{figure} Xampp is a web server that hosts PHP websites. Furthermore, XAMPP has a built-in MySQL database setup. It also has a GUI for MySQL. As a result, when you install XAMPP server on your PC, the MySQL database is also installed. Which means you won't need to install a separate MySQL database server. XAMPP is required for PHP websites, but not for JAVA. However, XAMPP is required for MySQL databases.\\
\begin{itemize}
\item Overleaf
\end{itemize}
\par \begin{figure}[h!]
\includegraphics[width=4cm, height = 2.5cm]{overleaf}
\centering
\end{figure} We decided to use Overleaf as the platform to complete our project documentation. It provides a collection of useful templates for various use cases that you can simply open in a new project, or download to use offline. Moreover, it also provides the convenience of an easy-to-use LaTeX editor with real-time collaboration and the fully compiled output produced automatically in the background as you type. The downside is the real-time collaboration feature is a premium feature so we will need to pay a few amount for it which is already stated in our cost estimation section (Refer Table II).\\
\begin{itemize}
\item Google Meet
\end{itemize}
\par \begin{figure}[h!]
\includegraphics[width=4cm]{googlemeet}
\centering
\end{figure} Like every other app development project, we would need a platform for meeting and discussing our development process. Hence, we decided to use Google Meet platform to conduct meetings every week and discuss about our current development, assignments and future plans. \\
\begin{itemize}
\item KakaoTalk
\end{itemize}
\par \begin{figure}[h!]
\includegraphics[width=4cm]{kakaotalk}
\centering
\end{figure} We decided to use the KakaoTalk chat platform to communicate with each other about meeting time, asking questions and sharing documents on the go. \\

\subsection{Software In Use}

1) Google Map API\\

\par Our application will have a map as its main backbone which will show up on the very main page after the user has logged in. So, we decided to use the Google Map API service to implement the map features into our application. The reason for this choice is the API key is not only free, but it is also Android Studio friendly; the Android studio provided various widgets and functions for the developers to manipulate the map such as adding markers, getting user current location, getting the distances between two places and much more.  \\

2) Social Media Login API\\
\par Our application will provide an alternate way for the users to log in; Google, Facebook and Twitter Login method. This way, users will not have to undergo the hassle of making a new account before using the application. Furthermore, users who logged in by these alternate methods will have their credentials handled by the respective services instead of our own server which grants them more security and safety.\\


\begin{table}[htbp]
\caption{Role Assignments}
\begin{center}
\begin{tabular}{ | m{4em} | m{6cm}| } 
  \hline
 \textbf{Name}& \textbf{Task}  \\
\hline
  \textbf{Megat} &   \\
  \hline
  \textbf{Wan} & \\ 
  \hline
  \textbf{Amin} & \\
  \hline
\end{tabular}
\end{center}
\end{table}

\section{Specifications}
\subsection{Account Creation and Login}

\\1) Login Page

Whenever a user first launch the application, they will be greeted by the login page of the application (refer Fig. 1). There will be a huge application name "DOOWA" on top of the page and under it will be 2 text spaces which will be used to receive input from the user for their username and password. For password textbox, the characters inputted will be privated and changes to dotted characters. Under these 2 text spaces, there will be a button to be clicked after the user filled their login credentials into the mentioned text spaces. Additionally, under the button, there will be 2 more smaller buttons, namely "Forgot password?" and "Register" which carries their specified task that are explained later .
\begin{figure}[h!]
\includegraphics[width=4cm]{./img/loginpage}
\centering
\caption{Login page; first page upon app execution.}
\end{figure}

\begin{itemize}
\item Empty username and password\\
When the "LOGIN" button is clicked while the username and password are empty, a warning will be displayed onto the respective text box which is currently empty and requires the user to fill up the text box to remove the warning.
\end{itemize}

\begin{figure}[h!]
\includegraphics[width=4cm]{./img/loginusernamefail}
\centering
\caption{Username textbox is empty.}
\end{figure}
\begin{figure}[h!]
\includegraphics[width=4cm]{./img/loginpasswordfail}
\centering
\caption{Password textbox textbox is empty.}
\end{figure}

\begin{itemize}
\item Both fields are filled and LOGIN button is clicked
\end{itemize}
When the user inputted both username and password text boxes and clicked the "LOGIN" button, the program will use the "startPut()" method provided by the JAVA Class "PutData" and send the inputted data to our server. Our server will then will check whether the combination is correct and then the result will be send back to the frontend side and inform the user whether the username and password combination is valid or not. While all of these process are ongoing, a circular loading bar will be shown at the center of the screen indicating the checking process in ongoing (refer Fig.4) . The loading bar will be gone after the checking process is completed successfully. After that, a message will appear the the bottom of the screen indicating whether the login is successful or not.\\

\begin{figure}[h!]
\includegraphics[width=4cm]{./img/loginloading}
\centering
\caption{Loading bar pops up while checking username and password.}
\end{figure}

\begin{itemize}
\item Wrong username and password combination
\end{itemize}
If the user inputted the wrong combination of username and password, after the LOGIN button is clicked, a message will appear at the bottom of the screen showing "Username or Password wrong" and the application will stay on the login page until the correct combination is inputted.

\begin{figure}[h!]
\includegraphics[width=4cm]{./img/loginfail}
\centering
\caption{Wrong username/password.}
\end{figure}

\begin{itemize}
\item Correct username and password combination
\end{itemize}
If the username and password provided by the user is valid, the user will be able to access the main page of the application. A message also will be displayed at the bottom of the screen saying "Logged In successfully!".

\begin{figure}[h!]
\includegraphics[width=4cm]{./img/loginsuccess}
\centering
\caption{Correct username/password.}
\end{figure}

\\2) User account registration\\
From the login page, there will be a button called "Register" which allows the user to create a new account for the first time. On the registration page, there will be 5 text fields for the user to input their details which are full name, username, email, password and password confirmation. Under the fields will be a button "Register" which will send all the inputted user details to the server when clicked. Under this button, at the bottom of the screen, there will be a text saying "Already have an account? Log in.". When users click this text, the application will simply send the user back to the previous login page.

\begin{figure}[h!]
\includegraphics[width=2.5 cm]{./img/registerpage}
\centering
\caption{Registration page.}
\end{figure}

\begin{itemize}
\item Full name
\end{itemize}
Full name can be any text. If this field is blank when the "Register" button is clicked, an error will pop up on the field. 
\begin{figure}[h!]
\includegraphics[width=3cm]{./img/fullnamefail}
\centering
\caption{Blank full name field error.}
\end{figure}

\begin{itemize}
\item Username
\end{itemize}
Username can be any text. If this field is blank when the "Register" button is clicked, an error will pop up on the field. 
\begin{figure}[h!]
\includegraphics[width=3cm]{./img/usernamefail}
\centering
\caption{Blank username field error.}
\end{figure}

\begin{itemize}
\item Email
\end{itemize}
Users must provide a valid email in this field. If this field is blank or invalid email address is used, an error will pop up on the field. \\

\begin{figure}[h!]
\includegraphics[width=3cm]{./img/emailfail1}
\centering
\caption{Blank email field error.}
\end{figure}

\begin{figure}[h!]
\includegraphics[width=3cm]{./img/emailfail2}
\centering
\caption{Invalid email error.}
\end{figure}

\begin{itemize}
\item Password and Confirmation
\end{itemize}
Users must provide a password that is at least 6 characters long. If this field is blank or invalid password is used, an error will pop up on the field. Same goes to the password confirmation field. If the string entered in both password and password confirmation field are different, an error will pop up on the password confirmation field.

\begin{figure}[h!]
\includegraphics[width=3cm]{./img/passwordfail1}
\centering
\caption{Blank password field error.}
\end{figure}

\begin{figure}[h!]
\includegraphics[width=3cm]{./img/passwordfail2}
\centering
\caption{Invalid password error.}
\end{figure}

\begin{itemize}
\item Successful Registration
\end{itemize}
Upon successful registration, user will be sent back to the login page of the application. Additionally, a message saying " User has been successfully created!" will be shown at the bottom of the screen. Now, user can use the credentials registered just now at the login page.

\begin{figure}[h!]
\includegraphics[width=4cm]{./img/registersuccess}
\centering
\caption{Successful registration.}
\end{figure}
\begin{itemize}
\item Failed Registration
\end{itemize}
Failed registration can only be occurred whenever a user tries to register using a username or email address that already has been registered. If such occurs, after clicking the "REGISTER" button, a message saying "Username/e-mail address already in use! Please try again." will be displayed at the bottom of the screen.

\begin{figure}[h!]
\includegraphics[width=4.3cm]{./img/registerfail}
\centering
\caption{Failed registration.}
\end{figure}

3) Sign In with Google Authentication\\

Additionally, users can also sign in into our application by using their existing Google account. Using this, users do not need to undergo the hassle process of creating new account for our application. If users uses this alternate sign in method, the user full name, username, and profile picture will be used instead for the account. With this method, users can feel more secure of their account and they are linked and administrated by the Google itself. To implement this feature, we used the Google Firebase authentication service and implemented the API into our application. Through the Firebase database, we can identify which users are logged in using the Google sign in method. Finally, users only need to sign in once through the Google authentication and the next time users open the app, they will be automatically signed in.


\begin{figure}[h!]
\includegraphics[width=4cm]{./img/googlesignin}
\centering
\caption{Google sign in button.}
\end{figure}



4) Password retrieval\\
\par Sometimes user might forgot their password from time to time. On the login page, there will be a small clickable text "Forgot Password?" which can be used by the users to reset their password. Users only need to provide their email address which was used for account registration into the field provided. Under the text field, a button "SEND EMAIL VERIFICATION" will be provided and whenever clicked, an email containing the hashed code of the user password will be sent to the inputted email. Failure will occur if the field is blank, invalid email format or the email is not found in the account database.

\begin{figure}[h!]
\includegraphics[width=3cm]{./img/retrievepage}
\centering
\caption{Password retrieval page.}
\end{figure}

\subsection{Map Services}

1) Implementing Map into Application\\

Whenever a user logged in into their account, the first thing that should be displayed on the screen is the map which will shows various markers indicating the locations of the people who are requesting donation. Therefore, the most important thing for our application would obviously be implementing map into our application. We decided to use the Google Maps API services because not only it is free, but is is also Android Studio friendly which will be providing various amount of functions and widgets that can be used to manipulate our map interface.\\

\begin{itemize}
\item Using Google Maps API
\end{itemize}
First we will go to https://console.cloud.google.com/ and click "Create Project".

\begin{figure}[h!]
\includegraphics[width=4cm]{./img/map1}
\centering
\caption{Create project.}
\end{figure}
After entering our project name, click the "Create" button and now our project is finally created in the Google console and ready to be provided with various APIs.\\
\begin{figure}[h!]
\includegraphics[width=4cm]{./img/map2}
\centering
\caption{Enter project name and create.}
\end{figure}
Then, on the search bar, search for "Maps SDK for Android" and click the "Enable" button to enable the API service for our project.
\begin{figure}[h!]
\includegraphics[width=4cm]{./img/map3}
\centering
\caption{Enable Maps SDK for Android.}
\end{figure}
Click on the "Credentials" from the tabs above and click on the "API key" to start creating the API key for our Google Maps. Later, API key will be successfully created and we can copy the key to be pasted into the project manifest file.
\begin{figure}[h!]
\includegraphics[width=4cm]{./img/map4}
\centering
\caption{Create API key.}
\end{figure}
\begin{figure}[h!]
\includegraphics[width=4cm]{./img/map5}
\centering
\caption{Copy API key and paste into project manifest file.}
\end{figure}
\\Finally, we can see the map being displayed in our application as the main display whenever user logs in.

\begin{figure}[h!]
\includegraphics[width=3cm]{./img/mapmain}
\centering
\caption{Main map display.}
\end{figure}
\begin{itemize}
\item Map Markers
\end{itemize}
\par On the map, there should be multiple markers that are indicating the location of other users which are requesting donations. These markers has different icons according to the donation type. If the donation type is "MONEY", then the marker should be replaced by money icon. The same rule will be applied for the other donation types. Whenever a marker is clicked, a label showing the details of the request such as the name of the requester, the type of donation needed and full address of the location should be displayed. If multiple markers are showing up at a very close distance, the markers should be grouped as one to make it easier for another user to click the overlapping markers. 

\begin{figure}[h!]
\includegraphics[width=3.5cm]{./img/mapmulti}
\centering
\caption{Multiple markers shown on map.}
\end{figure}
\begin{itemize}
\item Details upon clicking markers
\end{itemize}
\par As mentioned before, upon clicking any markers on the map, the application will bring the users to another page that will display all the details of the request such as the full name of the receiver, type of donation requested, time and date of request, and much more. On the page, there will be a call and message button which can be used by the donor to either call or message the receiver. Whenever these buttons are clicked, a pop up will appear at the bottom of the screen, showing multiple choices of applications which the donor prefer to make a call or send a message. Another button with the icon "navigation" also should be added on the details page. When this button is clicked, the application will launch the Google Maps application to show the location of the marker inside that app which will provide the navigation available to reach the location.\\

\begin{itemize}
\item Marker removed upon completed donation
\end{itemize}
\par On the requester side, whenever they are satisfied with the donation arrived, they can click the "Donation completed" button on their donation request page. Clicking this button will cause the marker of the request location to be removed from the map and database.\\

\subsection{Details Page}

\par As mentioned before, Whenever the users clicked the markers on the map, the details page of the donation request will appear. This details page will display all the necessary information of the requester such as their account username, profile picture, full name, address, explanation of the donation request, time and date and more.

\begin{figure}[h!]
\includegraphics[width=3.5cm]{./img/detailspage}
\centering
\caption{Details of donation request.}
\end{figure}
At the top of the page, the type of donation will be displayed. The type of donation can be either "Money", "Necessities", "Food" or "Others". Underneath it is the profile picture of the account of the requester. Besides it, there are two buttons for call and message. If any users clicked either of the button, the app will direct the user to another application accordingly.

\begin{figure}[h!]
\includegraphics[width=3.5cm]{./img/details1}
\centering
\caption{Buttons for contacting the requester.}
\end{figure}

\begin{itemize}
\item Call button
\end{itemize}
Whenever the users clicked this button on the details page, the application will immediately make call to the phone number provided by the requester.
\begin{itemize}
\item Message button
\end{itemize}
Whenever the users clicked this button on the details page, the application will immediately open up messaging app for the user to send message to the requester.\\

At the center of the details page will be filled by the details of the donation request as mentioned before. Additionally, at the bottom of the screen, the image uploaded by the requester will be shown. Moreover, there will also be another button "Transfer Money". With this button, any user will be directed to the banking app which will enable them to transfer money through online banking to the bank account number provided by the requester. Finally, at the very bottom of the page, users can click the "Report this request?" button if the request is inappropriate, suspicious or does not follow the donation request guidelines. Users can state the reason of the report and submit them.

\begin{figure}[h!]
\includegraphics[width=3.5cm]{./img/details2}
\centering
\caption{Buttons for transfer money and report.}
\end{figure}

\begin{figure}[h!]
\includegraphics[width=3.5cm]{./img/report}
\centering
\caption{Submission report page.}
\end{figure}

\subsection{Request Page}

Users can request their donation on this page. Users will be asked to fill up few details before proceeding to submit their donation request.

\begin{figure}[h!]
\includegraphics[width=3cm]{./img/rqPage}
\centering
\caption{Donation request page.}
\end{figure}

\begin{itemize}
\item "Donation Request" text
\end{itemize}

- This is the title of the page to indicate users that they are accessing the donation request page and its details.\\

\begin{itemize}
\item “Don't worry, Be Happy!” slogan
\end{itemize}
- This is the short slogan placed below the title page to motivate and desire users to be not ashamed for not being fortunate. \\

\begin{itemize}
\item Camera icon
\end{itemize}
- This lets users to upload an appropriate picture regarding their needs. There is only one picture can be uploaded at a time.\\

\begin{itemize}
\item Upload Picture text
\end{itemize}
- This is an indicator for users to notice where to upload picture regarding their needs. \\

\begin{itemize}
\item Requester Name section
\end{itemize}
-  Users must fill in this section with their correct full name. They may fill in this section with alphabets and numbers. The full name may not be too long exceeding the space provided.\\

\begin{itemize}
\item Choose Donation Type section
\end{itemize}
- Users can click this section to have a pop-down menu where they can go more details regarding what type of donation, they are giving by choosing the options such as "money", "necessities", "groceries", and "others". Users can select ‘others’ if the type of donation is not available among the given options. \\

\begin{figure}[h!]
\includegraphics[width=4cm]{./img/dropdown}
\centering
\caption{Donation type choices.}
\end{figure}

\begin{itemize}
\item Address section
\end{itemize}
- This lets users to fill in their full address of their home so that people will get the information of where they can provide the donation. In this section, users may fill in with alphabets, numbers, and also special characters. \\
 
\begin{itemize}
\item Details  section
\end{itemize}
- Users can add more details regarding their needs such as their necessities, how many family members are in need and the specific things that are needed. This section provides a space for users to add any more details that seem important for people to know that have not been informed in other sections. \\

\begin{itemize}
\item Available Meeting Time section
\end{itemize}
- This allows users to inform the time that is most suitable to provide donations or visitation by filling in when is the available time so that people can get aware and plan when the best time is to visit them. This field is optional.\\

\begin{itemize}
\item Contact Number section.
\end{itemize}
- This is the section for users to fill in their contact number in case of people want to reach out to them. This section lets user to fill in their contact number with numbers and special characters. It is preferable to have at least one contact number to be filled in.\\

\begin{itemize}
\item Bank and Account Number section
\end{itemize}
-Users can fill up this field with the details of their preferred bank account and the number. Using this will let other people to give donation through bank transfer without the hassle of meeting up. This field is optional.\\ 


\begin{itemize}
\item Set Map Location button 
\end{itemize}
- Users can press this button to navigate to the real map page where they can indicate the pinpoint on the map of where is exactly the location of their location. This helps to strengthen the address section where it lets people to confirm where the requester is by looking at the map. Clicking this button will bring the users to the map selection page to pin their selected location for their donation request. \\

\begin{itemize}
\item Guidelines
\end{itemize}
At the very bottom of the screen, there will be a link that will redirects the user to the guideline page on their defaul browser. Users can refer to this guidelines and take extra precautions so that the submission are complying with our submission guidelines.

\subsection{Food Bank Submission}

On this app, users can also open up a donation spot where other people who are in need can came by themselves to receive help needed. For that, user will be able to submit a submission for their open donation on the "Foodbank Submission" page. The food bank submission page is very similar to the donation request page with few different attributes.

\begin{figure}[h!]
\includegraphics[width=3cm]{./img/foodbankpage}
\centering
\caption{Main page for food bank submission.}
\end{figure}


\begin{itemize}
\item "Food Bank" text
\end{itemize}

- This is the title of the page to indicate organization (users) that they are accessing the food bank page and its details.\\

\begin{itemize}
\item “Let’s Distribute!” slogan
\end{itemize}

- This is the short slogan placed below the title page to motivate and desire users to be part of the food bank’s activity. \\
\begin{itemize}
\item Camera icon
\end{itemize}
- This lets organization (users) to upload an appropriate picture regarding their organizations. There is only one picture can be uploaded at a time.\\

\begin{itemize}
\item Upload Picture text
\end{itemize}
- This is an indicator for users to notice where to upload picture regarding their organizations. \\

\begin{itemize}
\item Organization Name section
\end{itemize}
- (Organization) Users must fill in this section with their correct organization’s name. They may fill in this section with alphabets and numbers. The organization’s name may not be too long exceeding the space provided.\\

\begin{itemize}
\item Choose Donation Type section
\end{itemize}
- Users can click this section to have a pop-down menu where they can go more details regarding what type of donation, they are giving by choosing the options such as money, necessities, groceries, and others. Users can select ‘others’ if the type of donation is not available among the given options. \\

\begin{itemize}
\item Address section
\end{itemize}
- This lets users to fill in their full address of the organization so that people will get the information of where they can provide the donation. In this section, users may fill in with alphabets, numbers, and also special characters. \\
 
\begin{itemize}
\item Details (tell me about what you are donating) section
\end{itemize}
- Users can add more details regarding their organizations such as their organization landmarks, person in charge and the process of providing donation to this organization. This section provides a space for organizations to add any more details that seem important for people to know that have not been informed in other sections. \\

\begin{itemize}
\item Operating Hours section
\end{itemize}
- This allows users to inform the time that is most suitable to provide donations by filling in when is the opening time and when is the closing time so that people can get aware and plan when the best time is to visit the organization.\\

\begin{itemize}
\item Contact Number section.
\end{itemize}
- This is the section for organization to fill in their contact number in case of people want to reach out the organization. This section lets user to fill in their contact number with numbers and special characters. It is preferable to have at least one contact number to be filled in.\\

\begin{itemize}
\item Set Map Location button 
\end{itemize}
- Users can press this button to navigate to the real map page where organization can indicate the pinpoint on the map of where is exactly the location of the organization. This helps to strengthen the address section where it lets people to confirm where the organization is by looking at the map. Clicking this button will bring the users to the map selection page to pin their selected location for their foodbank. \\

\begin{itemize}
\item Guidelines
\end{itemize}
At the very bottom of the screen, there will be a link that will redirects the user to the guideline page on their defaul browser. Users can refer to this guidelines and take extra precautions so that the submission are complying with our submission guidelines.


Unlike the Donation Request page, all the fields in the foodbank submission page are mandatory. Failure to such will cause the application to prompt an error to the blank fields until the user inputted the filed.

\begin{figure}[h!]
\includegraphics[width=4cm]{./img/fbError}
\centering
\caption{Error on blank field.}
\end{figure}

\subsection{Select Location Page}

This page will appear whenever a user clicked the "Set Map Location" from the donation request page or foodbank submission page. On this page, a map will be executed on the screen and there will be 2 button at the bottom of the screen which are "Set Location" and "Use Current Location". User can use this page to select their desired location for their request or foodbank to appear on the map at the Main page after a successful submission. There will be one draggable marker on the initial execution.

\begin{figure}[h!]
\includegraphics[width=4cm]{./img/mapPage}
\centering
\caption{Location selection page.}
\end{figure}

Before user are able to reach this page, they must turn on the location and network services on their devices. Failure to do such will causes an error and user will be rejected from entering the page.

\begin{figure}[h!]
\includegraphics[width=4cm]{./img/maperror}
\centering
\caption{Error on disabled location service.}
\end{figure}

\begin{itemize}
\item Set Location Button
\end{itemize}

If users wanted to use the draggable marker to mark their desired location, they should use this button. The position of the marker will then be recorded and be added to their submission. After clicking the button, another popup window will appear to get the confirmation of the user if they are really satisfied with the location. If they are satisfied, users should click "OK" prompt on the popup window. If not, users can click the "Cancel" button or just click anywhere outside the popup window to return to the location selection page.

\begin{figure}[h!]
\includegraphics[width=4cm]{./img/selectconfirm}
\centering
\caption{Confirmation popup window.}
\end{figure}
\begin{itemize}
\item Use Current Location
\end{itemize}
If users prefer to use their own current location, the users can click this button without having to drag the map marker. Using this will cause the donation request or foodbank submission to appear on the map on the exact current location of the user. User also must be aware that their location service should be on before clicking this button. Failure to do such will cause the app to return to the donation request page or foodbank submission age. After clicking the button, another popup window will appear to get the confirmation of the user if they are really satisfied with the location. If they are satisfied, users should click "OK" prompt on the popup window. If not, users can click the "Cancel" button or just click anywhere outside the popup window to return to the location selection page. 

\begin{figure}[h!]
\includegraphics[width=4cm]{./img/currconfirm}
\centering
\caption{Confirmation popup window.}
\end{figure}
If the user successfully chosen the desired location, a message will be shown at the bottom of the page informing that the submission is successfully uploaded and the marker should appear immediately on the map in the Map page.

\begin{figure}[h!]
\includegraphics[width=4cm]{./img/mapsuccess}
\centering
\caption{Successfully submitted the form.}
\end{figure}

\subsection{Account Page}

\par The application should have another page dedicated to show all the account information such as user name, full name, user ratings and others services and features. 

\begin{figure}[h!]
\includegraphics[width=4cm]{./img/accountpage}
\centering
\caption{User account page.}
\end{figure}

At the top of the page, the user profile picture account will be displayed and their user name and full name will be displayed underneath. On the right side, a numeric display will be shown to indicate the rating of the user. Below the user information is a rating bar that shows accordingly by the rating. If the user signed in using the Google authentication service, the profile picture, user name and full name will be displayed according to the user Google account.\\

\begin{figure}[h!]
\includegraphics[width=4cm]{./img/acc1}
\centering
\caption{User details are displayed at top.}
\end{figure}

Below the details are the 3 additional features that are provided for every user. \\
\begin{figure}[h!]
\includegraphics[width=4cm]{./img/acc2}
\centering
\caption{3 additional features.}
\end{figure}
\begin{itemize}
\item Request List
\end{itemize}
This tab will display all the request that has been submitted by the user. User can remove their made requests from here when they are satisfied with all the donation received. Doing such will remove the request from the list and also remove the marker from the map.
\begin{itemize}
\item Donation Tracker
\end{itemize}
This tab is only used for receiver who are going to receive donations via delivery or postage. In this tab, user can see the progress of the postage and their current status.
\begin{itemize}
\item Points Exchange
\end{itemize}
If the users completed any donations, they will receive points and those points can be exchanged to various prizes here. Such prizes are for example discount coupons, cashback, meals and much more.\\


At the half bottom of the page, we can see few services for the user to manage their account. 
\begin{figure}[h!]
\includegraphics[width=4cm]{./img/acc3}
\centering
\caption{Bottom half of page.}
\end{figure}
\begin{itemize}
\item Account Settings
\end{itemize}
As the name implies, this button will let the users to manage their account details such as full name, user name, change password, change profile picture, change app language and much more.
\begin{itemize}
\item Donation/Requests History
\end{itemize}
User can check their completed donations and deleted requests here for future references.
\begin{itemize}
\item FAQs
\end{itemize}
If users has any questions regarding our application features and services, they can refer to the frequently asked questions (FAQs) here. 
\begin{itemize}
\item Bugs Report
\end{itemize}
If users encountered any problem or error in our application they can submit a report form here. Additionally, they also can submit suggestion to improve the application.

\begin{figure}[h!]
\includegraphics[width=4cm]{./img/bugs}
\centering
\caption{Bugs report and suggestions page.}
\end{figure}

\begin{itemize}
\item Sign Out
\end{itemize}
Users can sign out from their account by just clicking this button. Upon click, a popup alert window will appear and ask the user if they really wanted to sign out. If "Cancel" is clicked, user will not sign out and stay on the account page. If "OK" button is clicked, user will be sent to the login page and logged out from the application.

\begin{figure}[h!]
\includegraphics[width=5cm]{./img/signout}
\centering
\caption{Sign out confirmation.}
\end{figure}
\section{Architecture and Design}

\textit{(TO BE CONTINUED...)}

\end{document}

